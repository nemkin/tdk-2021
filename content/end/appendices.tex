\appendix

\chapter*{\fuggelek}\addcontentsline{toc}{chapter}{\fuggelek}
\setcounter{chapter}{\appendixnumber}
%\setcounter{equation}{0} % a fofejezet-szamlalo az angol ABC 6. betuje (F) lesz
\numberwithin{equation}{section}
\numberwithin{figure}{section}
\numberwithin{lstlisting}{section}
%\numberwithin{tabular}{section}

\section{Első függelék}
\begin{figure}[!ht]
\centering
\includegraphics[width=150mm, keepaspectratio]{figures/TeXstudio.png}
\caption{A TeXstudio \LaTeX-szerkesztő.} 
\end{figure}

\clearpage

\section{Második függelék}
A Pitagorasz-tételből levezetve
\begin{align}
c^2=a^2+b^2=42.
\end{align}