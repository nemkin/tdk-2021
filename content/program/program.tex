\chapter{Simulator software}

In this chapter, I present the simulator software I wrote. I discuss the currently available solutions, why I chose to write the software, the architecture, the components and design patterns I used, the challenges I faced during the development and the solutions I found.

\section{Currently available solutions}

Since publicly accessible quantum computers currently only have around 5-10 qubits, it is not viable at the moment to run quantum walks on a real quantum computer. Hence why, when I started researching quantum walks, I quickly began looking into simulator software. While there are many of these currently available (for example \cite{Portugal} contains an extensive selection of open-source simulator software), most of them have at least one of the following issues:

\begin{enumerate}
\item Not maintained and developed anymore: the last commit was years ago.
\item Written in a low-level language, like C++, in a script-like fashion, with a prominent focus on memory and performance optimization while neglecting readability, modularity and extensibility.
\item Works exclusively on a specific type of graph, for example, n-dimensional lattices only.
\item Unable to compare and contrast classical and quantum walks on the same graph, running only quantum simulations.
\item Hard to understand as a novice.
\end{enumerate}

There is no general, open-source solution available that is designed and developed using sound software engineering practices and an architecture that allows for experimentation with different kinds of graphs with both classical and quantum simulations available.

I intend my solution to be valuable for research purposes while also providing a readable open-source codebase for college students to study the algorithm.

\section{Architecture}

The architecture of my simulator program employs the Strategy design pattern, which is described in the following way:

''Define a family of algorithms, encapsulate each one, and make them interchangeable. Strategy lets the algorithm vary independently from clients that use it.''~\cite{DesignPatterns}

\begin{figure}[H]
  \includegraphics[width=\linewidth]{./figures/program/strategy.png}
  \caption{UML diagram for the Strategy design pattern from~\cite{DesignPatterns}}
\end{figure}

This is a great design pattern for research purposes since it facilitates experimentation with various algorithms for the same purpose. It also makes the code easily readable, as the Strategy interface provides an abstraction layer between the Context and the concrete implementation.

\section{Language choice}

With the specified goals and the architecture in mind, I needed a language that is object-oriented, easily readable by beginners and has extensive capabilities for using complex numbers, linear algebra and plotting. For these purposes, I choose the Python language. Python is concise, it reads like pseudocode and has libraries such as NumPy, SciPy and Matplotlib, and so on, covering all areas of data science. Furthermore, it is well-known and extensively used by researchers with no software engineering background, allowing for easier collaboration.

\section{High level design}

The source code of the software can be divided into three parts:

\begin{itemize}
    \item Graph models
    \item Simulators
    \item Running, configuration and result collection
\end{itemize}

\change{TODO: UML diagram}

\subsection{Graph models}

I ran several experiments on various graphs while researching quantum graph walks, including paths, circles, bipartite graphs, hypercubes, and grids. Initially, I directly generated and stored their adjacency matrices, however, I quickly ran into memory scaling issues with this approach. Furthermore, in quantum research, graphs are typically built like 'Legos', glueing together a few common types, which was challenging to do with my original approach.

To combat these issues, I switched from the adjacency matrix representation to the oracle representation. The oracle is a function that returns the neighbours of a given vertex. Since I was using common graphs, I could calculate neighbouring indexes on the fly without storing anything about these graphs and only querying what is needed at the current step, dramatically reducing the memory requirements of the graph models.

\subsection{Simulators}

I implemented a classical and a quantum simulator class. The quantum simulator can currently simulate directed $k$ regular graphs, however since the permutation matrix decomposition, or in the undirected case, the edge coloring of the matrix is an NP-complete problem, in the current setup, the graph oracle must be implemented in a way that returns the neighbours in the same color order for all inputs. Since the human programmer designs the oracle, this is not a critical limitation at the moment. I have implemented a check as a safety guard to ensure the resulting shift matrices are unitary in case I made an error while coding one of the oracles.

\subsection{Running, configuration and result collection}

Using the above classes, I developed a framework in which experimental runs can be configured very quickly. The results of the run one are collected in an aggregated Latex document. It contains the given graph, the named type of the subgraphs, the adjacency matrices, the distribution results of the simulations and the eigenvalues and eigenvectors of the evolution operators. 

\change{Matlab scriptről is írjak?}